\documentclass[11pt]{article}
\usepackage[english]{babel}
\usepackage{array} % for better arrays (eg matrices) in maths
\usepackage{booktabs} %Tablas quedan mejor
\usepackage{vmargin} %M�rgenes
\usepackage{hyperref}
\usepackage[T1]{fontenc}
\setpapersize{A4}
\setmargins{2.5cm}       % margen izquierdo
{1.5cm}                        % margen superior
{16.5cm}                      % anchura del texto
{22.42cm}                    % altura del texto
{10pt}                           % altura de los encabezados
{1.5cm}                           % espacio entre el texto y los encabezados
{0pt}                             % altura del pie de p�gina
{2cm}                           % espacio entre el texto y el pie de p�gina

\usepackage[none]{hyphenat} %No corta palabras al final de las lineas
\usepackage{graphicx} %Im�genes
\usepackage{subfigure} % subfiguras
\graphicspath{{img/}} % Root for images
\usepackage{wrapfig}
\usepackage{siunitx} %Escribir n�meros y unidades
\usepackage{amsmath} %Mayor facilidad para escribir matrices y otros elementos matem�ticos
%\numberwithin{equation}{section} %En las ecuaciones se especifica tambi�n el cap�tulo
\usepackage{xcolor} %Colores (\color{})
\usepackage{lipsum}
\usepackage{topcapt}
\usepackage{parskip}
\usepackage{braket}
\usepackage{setspace}
\usepackage{listings}
\usepackage{fancyhdr} %Cabeceras
\pagestyle{fancy} % seleccionamos un estilo


\definecolor{mygreen}{rgb}{0,0.6,0}
\definecolor{mygray}{rgb}{0.5,0.5,0.5}
\definecolor{mymauve}{rgb}{0.58,0,0.82}


\lhead{SQL assignment} %CAMBIA
\rhead{Parallel and distributed systems} %CAMBIA





\begin{document}

\setstretch{1.3} % Line spacing of 1.5
\setlength{\parskip}{5mm}

\renewcommand{\tablename}{Table}

\thispagestyle{empty} %Primera p�gina sin n�mero
\begin{center}
	
	{\scshape\LARGE Parallel and distributed systems\par}
	\vspace{0.5cm}
	\rule{15cm}{0.8pt}\\
	{\huge\bfseries SQL assignment\par}
	\rule{15cm}{0.8pt}\\
	\vspace{0.5cm}
	{\large\itshape Mart� Municoy, Jorge Pardillos\par}

\end{center}
\setcounter{page}{1} %Empezamos a contar los n�meros



\pagenumbering{arabic} %Ponemos n�meros normales (�rabes) ya
\pagestyle{fancy} % seleccionamos un estilo


\textbf{Q0. Can you describe the series of steps to open a database for querying?}
\begin{enumerate}
\item Open mysql: \emph{mysql -u root -p} (and type your password).
\item Type \emph{show databases;} in order to explore the available databases.
\item Type \emph{use A;} in order to start querying the database ``A''.
\end{enumerate}


\textbf{Q1. What is the purpose of this query?\\ SELECT * from Sources;}

It shows the all the entries from the table called sources.

\textbf{Q2. Get 5 GenBank ids and corresponding descriptions.}

SELECT gbId, description FROM Descriptions LIMIT 5;

\textbf{Q3. What is the purpose of this query?\\ SELECT count(*) from LocusLinks;}

It gives the number of entries in the table LocusLinks;

\textbf{Q4. How many different Affy ids are in the expression data?}

SELECT count(*) FROM Data;

\textbf{Q5. What is the expression level of Affy id U95-32123\_at in experiment number 1?}

SELECT level FROM Data WHERE affyId=``U95-32123\_at''AND expt=1;

\textbf{Q6. Find all the gene descriptions, along with their GenBank ids containing the word ``Human''?}

SELECT * FROM Descriptions WHERE Descriptions LIKE ``\%Human\%'';

\textbf{Q7. What Gene Ontology descriptions (and corresponding accession) contain the phrase ``protein kinase''? Answer should be provided in ascending order of accessions.}

SELECT * FROM GO\_Descr WHERE description LIKE ``\%protein kinase\%'' ORDER BY goAcc ASC;

\textbf{Q8. Which AffyId of table Data correspond to sequences in Targets table with the phrase ``kinase'' in their description? Use the following command:\\ LOAD DATA INFILE `file.tsv' INTO TABLE Targets;\\ To add a new entry in Descriptions with the string ``kinase'' and the gbId= ``M18228''. Now repeat the query again}

SELECT Data.affyId\\
FROM Data, Targets, Descriptions\\
WHERE Data.affyId=Targets.affyId AND Targets.gbId=Descriptions.gbId AND Descriptions.description\\
LIKE ``\%kinase\%'';

This query gives 0 results.

We introduce one extra entry to the table descriptions: \emph{load data infile `new.tsv' into table Descriptions;}. The file is in the directory /var/lib/mysql/experiments/.

Repeating the same query again is still showing 0 results. There is an affyId corresponding to the gbId that we have introduced (M18228), but that affyId is not in the Data table.



\textbf{Q9. Get two affyId, uId and descriptions in LocusDescr in reverse alphabetical order of descriptions}

SELECT DISTINCT Data.affyId, UniSeqs.uId, UniDescr.description\\
FROM Data, UniSeqs, UniDescr, Targets\\
WHERE UniDescr.uId=UniSeqs.uId AND  UniSeqs.gbId=Targets.gbId AND Targets.affyId =Data.affyId\\
ORDER BY UniDescr.description DESC\\
LIMIT 2;

\textbf{Q10. How would you find the average expression level of each experiment in Data?}

SELECT exptId , AVG(level) FROM Data GROUP BY exptId;

\textbf{Q11. What is the average expression level of each array probe (affyId) across all experiments?}

SELECT affyId, AVG(level) FROM Data GROUP BY affyId;

\textbf{Q12. What is the purpose of the following query?\\
SELECT Data.affyId, Data.level, Data.exptId, DataCopy.affyId,\\
DataCopy.level, DataCopy.exptId\\
FROM Data, Data DataCopy\\
WHERE Data.level $>$ 10 * DataCopy.level\\
AND Data.affyId=DataCopy.affyId\\
AND Data.affyId LIKE ``AFFX\%''
LIMIT 10;}

From the table called ``Data'' it takes the entries with an affy-name beginning with the string ``AFFX'' and selects all of them for which the level number is 10 times bigger for an experiment than for some other experiments of the same affy-name.

\textbf{Q13. Write a query to provide three different descriptions for all gbId in table Targets}

There are four tables containing descriptions. Therefore there are 4 possible combinations and 4 possible outputs, depending on which three tables are chosen (Descriptions, LocusDescr, GO\_Descr or UniDescr).

One option will be:

SELECT Targets.gbId, Descriptions.description, LocusDescr.description, UniDescr.description\\
FROM Targets, Descriptions, LocusDescr, UniDescr, LocusLinks, UniSeqs\\
WHERE Targets.gbId=Descriptions.gbId AND Targets.gbId=LocusLinks.gbId\\
AND LocusLinks.linkId=LocusDescr.linkId AND Targets.gbId=UniSeqs.gbId\\
AND UniSeqs.uId=UniDescr.uId;

\newpage

\textbf{Q14. Write a query to provide all gene ontology (GO\_descr) descriptions related with all species in table Species sorted alphabetically and providing the first five results. Export the query to a tab-separated-file with the command:\\ SELECT * FROM TABLE INTO OUTFILE (`data.out');}

SELECT * FROM (\\
SELECT MY.species, GO\_Descr.description FROM GO\_Descr, (\\
SELECT LocusDescr.linkId AS linkId, LocusDescr.species FROM LocusDescr\\
UNION ALL\\
SELECT LocusLinks.linkId, Targets.species FROM LocusLinks, Targets\\
WHERE Targets.gbId=LocusLinks.gbId\\)
AS MY, Ontologies WHERE Ontologies.linkId=MY.linkId AND GO\_Descr.goAcc=Ontologies.goAcc ORDER BY species ASC LIMIT 5\\
) AS final INTO OUTFILE '/tmp/14.out';






\end{document}













